\documentclass[a4paper]{article}
\usepackage[margin=1in]{geometry}
\usepackage{blindtext}

\usepackage{fancyhdr}
\pagestyle{fancy}
\fancyhf{}
\rhead{\leftmark}
\lhead{Venture Capital}

\setcounter{section}{-1}

\title{LSE IRG\\
		Performance Competition\\
		Guidance Booklet}
\author{Cedric Tan - c.tan21@lse.ac.uk}
\date {2019-2020}

\begin{document}
\maketitle
{\small
  \noindent\textbf{Introduction to Competition}\\
  Introduction to what the IRG and the competition is trying to promote and engage you with. Overview of asset classes and how they might contribute to your trading strategy. \hspace*{\fill}[1]

  \vspace{10pt}
  \noindent\textbf{The Rule-Set}\\
  Looking at the rules of the competition and how trades will be executed on a weekly basis. What types of portfolios you will have and how you will trade individually or in groups to have the best possible outcome. \hspace*{\fill}[2]
   
  \vspace{10pt}
  \noindent\textbf{The Scoring}\\
  This section will go into how your level of return will be scored and how you will be placed on the ladder each week based on how well your portfolio is doing. Here, we want to maximise your alpha compared to the executive group who will hold an index fund. \hspace*{\fill}[3]

\newpage
\section{Introduction to Competition}
Welcome to IRG. If you've received one of these information packs (hard or soft copy), you are likely to be a recently inducted IRG analyst. First of all, congratulations on behalf of the executive team and myself. This packet will cover the asset classes which are involved in the IRG program and how they might be of aid when executing a trade in the performance competition.

\vspace{10pt}
\noindent The point of this competition is to promote interest in investment strategy whether you're aiming for short-term returns each week to accumulate your gains or utilising a long-term portfolio that has some sort of steady growth over the next two terms. This is for you to decide in the coming weeks. Note that discussion with your asset class heads is always welcome whether it be a certain stock, bond, currency, commodity, trend or style of investing, their knowledge might be able to help you execute a trade!

\subsection{Divisions}
Below is a brief introduction of the divisions at IRG:

\subsubsection{Equities}

\subsubsection{Currencies and Commodities}

\subsubsection{Fixed Income}

\subsubsection{Macro}

\subsubsection{Alternatives}
The Alternative Investment class covers most other types of investments that the other asset classes do not cover.

\vspace{10pt}
\noindent Listed below are some types of assets or investment vehicles that could be seen in Alternatives:
\begin{itemize}
	\item Hedge Funds
	\item Private Equity and Venture Capital
	\item Wine and Whiskey
	\item Vintage Cars
	\item Cryptocurrency
\end{itemize}
These are just some examples of the Alternative Investments that could be made and which attract some serious returns! Wine had outperformed a whole bunch of indices at the end of 2018 (FT 2019) and Private Equity, especially throughout Europe and more recently Spain, is making waves (FT 2019). Therefore, the breadth of scope within Alternatives is very, very large. In most cases, a lot of these types of investments do not have as much information as readily accessible, especially private equity, and takes a lot more time to scope out and recognise potential.

\vspace{10pt}
\noindent A piece of advice for alternatives is to pick a niche within alternatives, such as Cryptocurrency or Hedge Funds, and execute trades based on the knowledge you have in those areas. Some trades are inevitably harder, such as Private Equity, but if you believe that an Initial Public Offering (IPO) is going to be released with serious value following a previous Leveraged Buyout (LBO), it might be worth picking up in your portfolio.

\newpage
\section{The Rule-Set}
The Rule-Set is quite simple and is as follows:

\subsection{Teams}
You can participate in either teams or groups of up to 4. This is done to ensure that groups so large do not form which might over-complicate trades and execution each week.

\vspace{10pt}
\noindent	You can also capture teammates from across all asset classes! So if you have a friend who's in alternatives but you're in Fixed Income, the only barrier to them being on your team is their permission. We encourage this diversity to allow a greater flow of knowledge and more diverse trading outcomes.

\subsection{Trades}
To execute a trade the individual must be present or at least one member of your team has to be there to execute unless you have given us a good enough reason such that you or a sufficient enough number of your team cannot make the trade deadline. You are given the whole day before the IRG session to submit the trade with a team member required to confirm the trade during the evening session.

\vspace{10pt}
\noindent Additional rules are as follows:
\begin{itemize}
	\item Trades can be conducted to long or short a stock
	\item Alternative trades can be conducted as well subject to the heads approval
	\item Almost any type of instrument that can be tracked through a reliable financial medium can be traded such as options and other derivatives subject to the heads approval
\end{itemize}

\subsubsection{Long Trade}
Simple acquisition of asset e.g. stock, bond, currency that is placed into your portfolio (long-term or short-term). This will most likely be the majority of confident trades.

\subsubsection{Short Trade}
Acquisition of shorting shares which will require a certain margin to be deposited with the \textbf{broker} as in a real trade. The margin will be calculated from week to week.

\vspace{10pt}
Naturally, if you go into debt, interest might accumulate compounded on a weekly basis which might put you further into debt unless you manage to recuperate your losses through gains elsewhere in your portfolio.

\subsubsection{Alternative Trades}
Alternative trades may need to be discussed with section heads to see if they can be implemented into the game. An example of this could be a \textbf{butterfly spread} which is a complex method of trading options. However, such a trading strategy may be too complicated to implement in the Performance Competition.

\vspace{10pt}
\noindent \textbf{It is highly recommended that analysts approach the game by using their fundamental analysis of companies.} Further to that, a \textbf{catalyst} may be necessary for these stocks to accelerate in growth or decline. \textbf{Analysts should look for companies that exhibit these qualities to make a gain over two school terms or a week!} We encourage all to invest and not hold cash, however, a legitimate strategy could be to hold a commodity that might rise in value over time due to a stimulus e.g. Gold and Inflation.

\subsection{Portfolios}
This section will tackle a few things such as funds, long-term and short-term portfolios, windfalls and how you might be able to access these portfolios to track performance.
\subsubsection{Funds}
\begin{itemize}
	\item Each team will start with \$100,000
	\item This is irrespective of the number of people within your team
	\item You will be able to convert funds at the exchange rate on the day if desired
	\item You are not allowed to take credit lines to repay debts or margin calls
\end{itemize}

\subsubsection{Long-term}
If a division is made for a long portfolio:
\begin{itemize}
	\item Trades are limited in the portfolio to two a week
	\item However, no brokerage fees are applied
\end{itemize}

\subsubsection{Short-term}
If a division is made for a short portfolio:
\begin{itemize}
	\item Trades are unlimited in the portfolio each week
	\item However, a brokerage fee, although minimal, will be applied
\end{itemize}

\subsubsection{Windfalls}
To incentivise trading, there will be a windfall tax for sitting on cash above \$20,000. This is to ensure that all competitors are engaging in trading activity and picking stocks which they are at least confident in.

\newpage
\section{The Scoring}
The Scoring will be a simple benchmarking system that each portfolio from each team (or individual) will undergo. The IRG heads will collectively hold an indexed fund dependent on what stocks or exchanges everyone will be using.

\subsection{Return}
The first part of scoring will be dedicated to measuring the return of your overall portfolio. This is a return on the \$100,000 initially given at the start of the competition. Return overall will be the most important factor for the end of the week and subsequently at the end of the competition.

\subsection{Ranking}
Ranking will be done each week to show performance of competitors. Ranking will be shown before trades are executed on Wednesday with their portfolios also briefly disclosed to see if competitors might want to best the others in some way of adapting one another's strategy.

\vspace{10pt}
\noindent This ranking allows for competitors to see how their strategy is faring compared to others. Although you might be looking at some good returns, other teams might be looking at returns which are far larger. Rankings are an opportunity to see what you can do better.

\subsection{Overall}
The final ranking will take place closer to the end of Lent Term. The overall result will be purely based on returns throughout the whole yearThis will also be followed by a prize-giving ceremony to the team or individual which performed the best in the Performance Competition.

\newpage
\section{Final Notes}
We encourage you to set up your own portfolio without the limitations imposed by the IRG Performance Competition rules so that you can experiment different styles of trading across the two terms.

\vspace{10pt}
\noindent It is highly regarded that simplicity, in most cases, can earn the highest returns by selecting a few good companies that you are confident in whether they have large market capitalisations or not or companies you believe will truly fail. Risky options can pay off as well such as trading options in high volume and high margin, however, the risk, especially with no chance of obtaining further credit, may catapult your portfolio into a disastrous position.

\vspace{10pt}
\noindent Remember that this game is supposed to be \textbf{simple} and although you can execute some complex trades, your own explanation and how you will want to go about it has to be explained \textbf{fully} to the IRG heads such that they can aid you in completing the trade on the excel spreadsheet.

\vspace{10pt}
\noindent If you have any further enquiries, contact a member of the IRG team to talk further. If there are questions specific to the Performance Competition, it would be advisable to speak to Cedric Tan - Head of Alternatives or send him an email through the contact on the front matter.

\vspace{30pt}
\begin{center}
	\huge{Good Luck!}
\end{center}

\end{document}