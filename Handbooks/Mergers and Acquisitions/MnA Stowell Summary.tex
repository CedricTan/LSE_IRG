\documentclass[10pt, a4paper]{article}
\usepackage[margin=1in]{geometry}
\usepackage[bottom]{footmisc}
\usepackage{blindtext}
\usepackage{graphicx}
\graphicspath{ {VC_img/} }


\usepackage{fancyhdr}
\pagestyle{fancy}
\fancyhf{}
\rhead{\leftmark}
\lhead{M\&A}

\setcounter{section}{-1}

\title{LSE IRG\\
		Mergers and Acquisitions - Stowell Summary}
\author{Cedric Tan - c.tan21@lse.ac.uk}
\date {2019-2020}

\begin{document}
\maketitle

\begin{abstract}
	Simple summary of Mergers and Acquisitions ("\textbf{M\&A}") taken from Stowell's Investment Banks, Hedge Funds and Private Equity. It covers the core of M\&A, synergies, methods of acquisition along with some analytical tools that are commonly used in the process.
\end{abstract}

\newpage
\pagestyle{fancy}

\noindent \textbf{Core of M\&A}\\
It is the \textbf{buying and selling of corporate assets} in order to achieve one or more strategic objectives. Before entering into an acquisition, companies typically compare the costs, risks and benefits of an acquisition with their organic opportunity. This is what is called \textbf{Greenfield Analysis} or more simply put, \textbf{Buy versus Build}.

\vspace{5pt}
\noindent The inverse decision of whether or not to sell is about analysing the benefits of continuing to operate the business itself or if taking the risk-adjusted option of monetizing the asset is better to look for other opportunities. \hspace*{\fill} [1]

\vspace{10pt}
\noindent \textbf{Creating Value}\\
This is a contentious point of M\&A as there is ongoing debate on whether or not, apart from enriching investment bankers, that the process of merging and acquiring businesses actually creates value which is beneficial to the company's shareholders.

\vspace{5pt}
\noindent To gauge whether or not value is added, \textbf{comparative analysis} is usually done whereby share prices of other companies in the same industry over the same interval of time are compared. The share price usually is indicative of the value change, up being more beneficial and down being detrimental. \hspace*{\fill} [2]

\vspace{10pt}
\noindent \textbf{Strategic Rationale}\\
The strategic rationale can usually be a result of the desire for some sort of synergy that arises from the merge or acquisition.

\vspace{5pt}
\noindent \textit{Cost Synergies:} are most important and they arise through efficiencies created from the elimination of redundant activities. These can be identified in the following areas: Admin (central back office functions); Manufacturing (eliminating overcapacity); Procurement (purchasing power benefits through pooled purchasing); Marketing and Distribution (cross-selling and using common sales channels along with consolidated warehousing); and R\&D (eliminating R\&D overlap in personnel and projects).

\vspace{5pt}
\noindent \textit{Revenue Synergies:} are less important as they are more unpredictable to forecast and plan for. These come as a result of combining revenues from two companies with the ability to cross-sell or up-sell existing customers through synergies that the companies will have in common. McKinsey states that 88\% of acquirers were able to capture at least 70\% of estimated cost savings while only 50\% of acquirers were able to capture 70\% of estimated revenue synergies.

\vspace{10pt}
\noindent \textbf{Credit Ratings}\\
Something to consider is the credit rating of the company as a result of a merge or acquisition. A transaction can result in any movement in the credit rating whether it be up, down or equal. Share-based acquisitions have a more salutary effect on the acquirer's balance sheet and so ratings may not be as negatively impacted compared to a cash-based acquisition.

\vspace{10pt}
\noindent \textbf{Regulatory Considerations}\\
Companies and their legal and investment banking advisors must analyse the regulatory approvals that are necessary for transactions to occur. This depends on country to country so you will need to understand the environment in which you are situated.

\vspace{10pt}
\noindent \textbf{Social and Constituent Considerations}\\
Some social issues should be considered when conducting a merger or acquisition. Examples could be: what is the quality of the target company's management team and should they be retained or asked to leave? Could two different management teams work together without disrupting business? Are there significant relocation or tax-based issues that the team has to consider? These social issues can arise at the heart of an M\&A transaction.

\vspace{5pt}
\noindent Principal constituents that must be considered in any potential transaction include:
\begin{enumerate}
	\item Shareholders: concerned about valuation, control, risk and tax
	\item Employees: who focus on compensation, termination risk and benefits
	\item Regulators: who must be persuaded that antitrust, tax and securities laws are adhered to
	\item Union Leaders: who worry about job retention and seniority issues
	\item Credit Rating Agencies: who focus on credit quality issues
	\item Equity Research Analysts: who focus on growth, margins, market share, EPS etc.
	\item Debt Holders: who consider whether debt will be increased or retired or if there is potential for changing debt values
\end{enumerate}

\vspace{10pt}
\noindent \textbf{Investment Banker Role}\\
Investment bankers identify potential companies or divisions to be bought, sold, merged or joint-ventured. This is done by creating scenarios for successful transactions to analyse synergies and to deliver a fairness opinion. This fairness opinion reviews a deal's valuation on standard valuation processes to see whether or not the venture does not cheat shareholders, buyers or sellers.

\vspace{10pt}
\noindent \textbf{Types of Acquisitions}\\
There are three main types of ways in which a company can be acquired:
\begin{enumerate}
	\item A merger
	\item An acquisition of stock directly form the target company shareholders
	\item An acquisition of the target company assets
\end{enumerate}
\vspace{5pt}
\noindent The third option is rarely used since it is usually tax-inefficient so the first two methods are summarised in the following:

\vspace{5pt}
\noindent \textit{Merger}\\
This is the legal combination of two companies based on either a stock swap or cash payment to the target company shareholders. Shareholders have to vote at least 50\% in favour of the merger to occur. Higher percentages may be company specific based on the information on corporate practices within the company itself. Typically the acquiring firm has control of the board and senior management but in a merger of equals ("\textbf{MOE}"), the controlling factor is less certain. Usually one side or the other is \textbf{subtly dominant.}

\vspace{5pt}
\noindent \textit{Tender Offer}\\
A tender offer can also occur, this is the public purchase of shares without the need for a shareholder vote and is simply based on market supply to be purchased. This is in a constrained time period that the acquiring company publicises. Typically a tender offer is initiated when the target company's shareholders are not in support of a merger and so can be completed faster than a merger.

\vspace{10pt}
\noindent \textbf{Breakup Fee}\\
This fee is paid if a transaction is not completed because a target company walks away from the transaction after a merger or stock purchase agreement. This fee is designed to discourage other firms from making bids for the target company since they would, in effect, end up paying the breakup fee is successful in their bid. 

\vspace{10pt}
\noindent \textbf{Alternative Sale Processes}\\
Sell side deals are more likely to be completed than buy side deals so these are usually prioritised. Below are a few sales processes that can be executed.

\vspace{5pt}
\noindent \textit{Preemptive}\\
Bankers screen and identify the single most likely buyer and contact this buyer only. This maximises confidentiality and speed but may reduce the potential for profit.

\vspace{5pt}
\noindent \textit{Targeted Solicitation}\\
Bankers identify and contact the two to five most likely buyers. This aims to maximise confidentiality and speed but also increase the amount of profit beyond a preemptive approach.

\vspace{5pt}
\noindent \textit{Controlled or Limited Auction}\\
Bankers approach between six to twenty buyers who have been prescreened to be the most logical. The process is slower and becomes known to the market much faster but the potential for gain is much higher.

\vspace{5pt}
\noindent \textit{Public Auction}\\
The company publicly announces the sales process and invites all interested parties to participate. This creates disruptions in the company's business compared to a control auction and can take much more time. However, the potential for profit is the greatest out of all of them.

\end{document}