\documentclass[10pt, a4paper]{article}
\usepackage[margin=1in]{geometry}
\usepackage[bottom]{footmisc}
\usepackage{blindtext}
\usepackage{graphicx}
\graphicspath{ {VC_img/} }


\usepackage{fancyhdr}
\pagestyle{fancy}
\fancyhf{}
\rhead{\leftmark}
\lhead{M\&A}

\setcounter{section}{-1}

\title{LSE IRG\\
		Mergers and Acquisitions - Stowell Summary}
\author{Cedric Tan - c.tan21@lse.ac.uk}
\date {2019-2020}

\begin{document}
\maketitle

\begin{abstract}
	Simple summary of Mergers and Acquisitions ("\textbf{M\&A}") taken from Stowell's Investment Banks, Hedge Funds and Private Equity. It covers the core of M\&A, synergies, methods of acquisition along with some analytical tools that are commonly used in the process.
\end{abstract}

\newpage
\pagestyle{fancy}

\noindent \textbf{Core of M\&A}\\
It is the \textbf{buying and selling of corporate assets} in order to achieve one or more strategic objectives. Before entering into an acquisition, companies typically compare the costs, risks and benefits of an acquisition with their organic opportunity. This is what is called \textbf{Greenfield Analysis} or more simply put, \textbf{Buy versus Build}.

\vspace{5pt}
\noindent The inverse decision of whether or not to sell is about analysing the benefits of continuing to operate the business itself or if taking the risk-adjusted option of monetizing the asset is better to look for other opportunities. \hspace*{\fill} [1]

\vspace{10pt}
\noindent \textbf{Creating Value}\\
This is a contentious point of M\&A as there is ongoing debate on whether or not, apart from enriching investment bankers, that the process of merging and acquiring businesses actually creates value which is beneficial to the company's shareholders.

\vspace{5pt}
\noindent To gauge whether or not value is added, \textbf{comparative analysis} is usually done whereby share prices of other companies in the same industry over the same interval of time are compared. The share price usually is indicative of the value change, up being more beneficial and down being detrimental. \hspace*{\fill} [2]

\vspace{10pt}
\noindent \textbf{Strategic Rationale}\\
The strategic rationale can usually be a result of the desire for some sort of synergy that arises from the merge or acquisition.

\vspace{5pt}
\noindent \textit{Cost Synergies:} are most important and they arise through efficiencies created from the elimination of redundant activities. These can be identified in the following areas: Admin (central back office functions); Manufacturing (eliminating overcapacity); Procurement (purchasing power benefits through pooled purchasing); Marketing and Distribution (cross-selling and using common sales channels along with consolidated warehousing); and R\&D (eliminating R\&D overlap in personnel and projects).

\vspace{5pt}
\noindent \textit{Revenue Synergies:} are less important as they are more unpredictable to forecast and plan for. These come as a result of combining revenues from two companies with the ability to cross-sell or up-sell existing customers through synergies that the companies will have in common. McKinsey states that 88\% of acquirers were able to capture at least 70\% of estimated cost savings while only 50\% of acquirers were able to capture 70\% of estimated revenue synergies.

\vspace{10pt}
\noindent \textbf{Credit Ratings}\\
Something to consider is the credit rating of the company as a result of a merge or acquisition. A transaction can result in any movement in the credit rating whether it be up, down or equal. Share-based acquisitions have a more salutary effect on the acquirer's balance sheet and so ratings may not be as negatively impacted compared to a cash-based acquisition.

\vspace{10pt}
\noindent \textbf{Regulatory Considerations}\\
Companies and their legal and investment banking advisors must analyse the regulatory approvals that are necessary for transactions to occur. This depends on country to country so you will need to understand the environment in which you are situated.

\vspace{10pt}
\noindent \textbf{Social and Constituent Considerations}\\
Some social issues should be considered when conducting a merger or acquisition. Examples could be: what is the quality of the target company's management team and should they be retained or asked to leave? Could two different management teams work together without disrupting business? Are there significant relocation or tax-based issues that the team has to consider? These social issues can arise at the heart of an M\&A transaction.

\vspace{5pt}
\noindent Principal constituents that must be considered in any potential transaction include:
\begin{enumerate}
	\item Shareholders: concerned about valuation, control, risk and tax
	\item Employees: who focus on compensation, termination risk and benefits
	\item Regulators: who must be persuaded that antitrust, tax and securities laws are adhered to
	\item Union Leaders: who worry about job retention and seniority issues
	\item Credit Rating Agencies: who focus on credit quality issues
	\item Equity Research Analysts: who focus on growth, margins, market share, EPS etc.
	\item Debt Holders: who consider whether debt will be increased or retired or if there is potential for changing debt values
\end{enumerate}

\vspace{10pt}
\noindent \textbf{Investment Banker Role}\\
Investment bankers identify potential companies or divisions to be bought, sold, merged or joint-ventured. This is done by creating scenarios for successful transactions to analyse synergies and to deliver a fairness opinion. This fairness opinion reviews a deal's valuation on standard valuation processes to see whether or not the venture does not cheat shareholders, buyers or sellers.

\vspace{10pt}
\noindent \textbf{Types of Acquisitions}\\
There are three main types of ways in which a company can be acquired:
\begin{enumerate}
	\item A merger
	\item An acquisition of stock directly form the target company shareholders
	\item An acquisition of the target company assets
\end{enumerate}
\vspace{5pt}
\noindent The third option is rarely used since it is usually tax-inefficient so the first two methods are summarised in the following:

\vspace{5pt}
\noindent \textit{Merger}\\
This is the legal combination of two companies based on either a stock swap or cash payment to the target company shareholders. Shareholders have to vote at least 50\% in favour of the merger to occur. Higher percentages may be company specific based on the information on corporate practices within the company itself. Typically the acquiring firm has control of the board and senior management but in a merger of equals ("\textbf{MOE}"), the controlling factor is less certain. Usually one side or the other is \textbf{subtly dominant.}

\vspace{5pt}
\noindent \textit{Tender Offer}\\
A tender offer can also occur, this is the public purchase of shares without the need for a shareholder vote and is simply based on market supply to be purchased. This is in a constrained time period that the acquiring company publicises. Typically a tender offer is initiated when the target company's shareholders are not in support of a merger and so can be completed faster than a merger.

\vspace{10pt}
\noindent \textbf{Breakup Fee}\\
This fee is paid if a transaction is not completed because a target company walks away from the transaction after a merger or stock purchase agreement. This fee is designed to discourage other firms from making bids for the target company since they would, in effect, end up paying the breakup fee is successful in their bid. 

\vspace{10pt}
\noindent \textbf{Alternative Sale Processes}\\
Sell side deals are more likely to be completed than buy side deals so these are usually prioritised. Below are a few sales processes that can be executed.

\vspace{5pt}
\noindent \textit{Preemptive}\\
Bankers screen and identify the single most likely buyer and contact this buyer only. This maximises confidentiality and speed but may reduce the potential for profit.

\vspace{5pt}
\noindent \textit{Targeted Solicitation}\\
Bankers identify and contact the two to five most likely buyers. This aims to maximise confidentiality and speed but also increase the amount of profit beyond a preemptive approach.

\vspace{5pt}
\noindent \textit{Controlled or Limited Auction}\\
Bankers approach between six to twenty buyers who have been prescreened to be the most logical. The process is slower and becomes known to the market much faster but the potential for gain is much higher.

\vspace{5pt}
\noindent \textit{Public Auction}\\
The company publicly announces the sales process and invites all interested parties to participate. This creates disruptions in the company's business compared to a control auction and can take much more time. However, the potential for profit is the greatest out of all of them.

\newpage
\noindent \textbf{Corporate Restructuring}\\
M\&A focuses on strategic opportunities that unlock shareholder value through the separation of a subsidiary from a parent company. Separating business can improve operating performance, reduce risk profiles (including credit risk), and provide more efficient access to public capital markets. Separations can either be private or public. Private involves selling a subsidiary to private investors or to another company. A public market event involves selling or separating part of or the entire subsidiary in a public market transaction such as an IPO, carve-out, spin-off, split-off or tracking stock transaction.

\vspace{10pt}
\noindent \textit{IPO}\\
A subsidiary IPO is the sale of all shares of a subsidiary to new public market shareholders in exchange for cash. This creates a new company with a new stock that trades independently from the former parent company stock. If the cash received by the praent is in excess of the parent's tax basis, then the IPO is a taxable event for the parent.

\vspace{10pt}
\noindent \textit{Carve-out}\\
The sale through an IPO of a portion of the shares of a subsidiary to new public market shareholders in exchange for cash is called a carve-out. This leaves the parent with some ownership of the subsidiary with usually less than 20\% of the subsidiary sold in the carve-out. One consideration is the potential conflict the subsidiary and the parent company might have, for example, if they were vertically integrated, supply chain considerations may come into conflict.

\vspace{10pt}
\noindent \textit{Spin-Off}\\
In a spin-off, the parent gives up control over the subsidiary by distributing subsidiary shares to parent company shareholders. This is effectively just a redistribution of assets owned by shareholders to themselves but under a new share name. A spin-off provides the new company with its own acquisition currency, enables the new company management to receive incentive compensation and unlocks the value of the business if comparable companies trade at higher multiples than the parent company multiple. That is to say, it drops the deadweight of the parent company that impacted value of the subsidiary.

\vspace{10pt}
\noindent \textit{Split-Off}\\
In a split-off, the parent company delivers ahres of the subsidiary to only those parent shareholders who are willing to exchange their parent company shares for the shares of the subsidiary. This is preferred when parent company shareholders prefer holding one share type over another i.e. the parents company over the subsidiary or vice-versa. Sometimes a premium must be considered in the exchange due to the ultimatum given by the company. These are less common than a spin-off.

\vspace{10pt}
\noindent \textit{Tracking Stock}\\
In a tracking stock transaction, a separate class of parent company shares is distributed to existing shareholders of the parent company either through a spin-off or through a sale to new shareholders in a carve-out. This complicates corporate governance as there is no formal legal separation and a single board of directors continues to operate both businesses. In addition, both entities are liable for one another's debt obligations. This is a confusing form of transaction with its logic constantly debated.

\vspace{10pt}
\noindent \textbf{Risk Arbitrage}\\
In a stock for stock acquisition, some traders will buy the target company's stock and simultaneously short the acquiring company's stock. This purchase is motivated by the fact that after announcement of a pending acquisition, the target company's share price typically trades at a lower price in the market compared to the price reflected by the exchange ratio that will apply at the time of closing. Traders who expect that the closing will eventually occur can make profits by buying the target company's stock and then receiving the acquiring company's stock at closing. \textit{Investment bankers keep close track of risk arbitrage activity throughout the transaction period.} This activity may impact the price of a merger or acquisition significantly. 

\vspace{10pt}
\noindent \textbf{Valuation}\\
There are four basic valuation methods that determine the appropriate price for acquisition of a controlling interest in a company, namely:
\begin{enumerate}
	\item Comparable Company Analysis
	\item Comparable Transaction Analysis
	\item Leveraged Buyout ("\textbf{LBO}") Analysis
	\item Discounted Cash Flow ("\textbf{DCF}") Analysis
\end{enumerate}
In addition, a sum of parts analysis may be used if a company has many different businesses i.e. conglomerates, with the idea that individual businesses sold through carve-outs could create shareholder value in excess of the company's current share price. The key to selecting the best methodology is contextual so it is based on the industry, available information and market precedent.

\vspace{5pt}
\noindent Comparable company analysis and comparable transactions analysis are multiples-based methods for determining value in relation to a set of peers. This is calculated on a metric such as earnings or earnings before interest, taxes, depreciation and amortisation ("\textbf{EBITDA}"). EBITDA is generally used as a proxy for cash flow but they are not identical. To use cash flow would mean looking into the balance sheet as well as the income statement whilst EBITDA can be taken from the income statement alone. The most common multiples are Enterprise Value to EBITDA ("\textbf{EV/EBITDA}"), Price to Earnings ("\textbf{P/E}"), and Price to Book  ("\textbf{P/B}").

\vspace{5pt}
\noindent DCF and LBO analysis are cash flow-based methods of valuation. Both require projected future cash flows which are discounted by a company's cost of capital. A DCF analysis attempts to determine the intrinsic value of a company based on future cash flow projections. An LBO analysis attempts to determine an internal rate of return ("\textbf{IRR}") for a private equity firm acquirer based on future cash flow projections. The challenge is developing accurate models since 5-10 years of cash flow is industry convention for this valuation method.

\vspace{5pt}
\noindent Below are deeper dives into each method as presented by Stowell:

\vspace{10pt}
\noindent \textit{Comparable Company Analysis}\\
Comparable Company Analysis provides a helpful reference point but is not used as a principal basis for determining the value for an acquisition target since it does not incorporate a control premium. This analysis relies on the assumption that markets are efficient and current trading values are an accurate reflection of current industry trends, business risks, growth prospects and so forth. 

\vspace{5pt}
\noindent Comparable companies in many cases can be analysed based on their P/E multiple which is calculated by diving the current stock price by the annual earnings per share. Comparison of P/E ratios with other companies can give insight into whether or not a company is under, over or equally valued, taken at face value. Comparable companies should also be analysed based on their enterprise value which represents the total cost of buying a company. This is equal to the current market value of equity plus net debt (and minority interests if they exist). Because EV takes into consideration the value of equity and net debt, it provides a better comparison across companies with differing capital structures. Hence, using the EV/EBITDA multiple is part of the valuation method.

\vspace{10pt}
\noindent \textit{Comparable Transactions Analysis}\\
This focuses on M\&A transactions in which comparable companies were acquired. A comparable transaction analysis also includes control premiums (and expected synergies) and so multiples will generally be higher than for comparable companies and more reflective of a reasonable price for an acquisition. For example, if a company was acquired for an EV/EBITDA multiple of 10x to 11x, then this multiple range should also be applied to similar targets within the same industry.

\vspace{5pt}
\noindent Comparable transactions are typically drawn from the previous five to ten-year period, although the most recent are considered the most representative. If done properly, they can help determine a range of potential prices to offer or propose when purchasing or selling a company. After completing this process, it is important to triangulate by completing two other processes of valuation.

\vspace{10pt}
\noindent \textit{Discounted Cash Flow Analysis}\\
A DCF analysis is considered an essential valuation methodology since it attempts to determine the intrinsic value of a company. DCF relies on the projected cash flows of the company. A DCF analysis assumes that the value of a company (the enterprise value) is equal to the value of its future cash flows discounted by the time value of money and the riskiness of those cash flows. The company's value is calculated in two parts in a DCF analysis: (1) the sum of the cash flows during the projection period and (2) the terminal value (the estimated value of the business at the end of the projection period). Both parts are discounted using the company's weighted average cost of capital ("\textbf{WACC}"). The end result is determination of the net present value ("\textbf{NPV}") of the company's operating assets.

\vspace{5pt}
\noindent In a DCF analysis, future projections can incorporate changes in a company's long-term strategic plan. As a result, a DCF analysis is flexible enough to incorporate changing assumptions about growth rates and operating margins, wile allowing for adjustments for non-operating items. However, it also has its limitations such as being overly dependent on accurate projections. The longer the projection, the less confident one should be about the analysis. It is important in a DCF analysis to project cash flows through the period of time covered by a full operating cycle so that cash flows at the end of the projection period are "normalised". This is usually called the terminal value ("\textbf{TV}") of company. There are two methods to calculating it:
\begin{itemize}
	\item Terminal multiple method which applies a multiple such as EV/EBITDA to projected EBITDA at the TV date
	\item The perpetuity growth rate method which is determined based on the formula: $TV = FCF * (1+g)/(r-g)$, where FCF is free cash flow, r is equal to WACC and g is perpetual growth equal to the expected rate of inflation and long-term GDP growth.
\end{itemize}
From this, we can state the three necessary steps to complete a DCF valuation:
\begin{enumerate}
	\item Determine unlevered free cash flows for an up to 10-year period such that the end of this period represents a steady state condition for the company i.e. running consistently
	\item Estimate the terminal value of the company at the time when the company has reached a steady state (which coincides with the end of the cash flow forecast period) and continuing into perpetuity
	\item Determine WACC, which is the blended cost of debt and equity for the company, and then discount the unlevered free cash flows and the terminal value by WACC to create a present value (enterprise value) of the company
\end{enumerate}

\vspace{10pt}
\noindent \textit{Leveraged Buyout Analysis}\\
A leveraged buyout ("\textbf{LBO}") analysis is a relevant acquisition analysis when there is the possibility of a financial sponsor buyer. Financial sponsors are private equity firms that purchase companies using equity they have raised in a private investment fund combined with new debt raised to facilitate the purchase. Usually, a large amount of debt is used to fund the acquisition. Targets are typically companies in mature industries that have stable and growing cash flow in order to service large debt obligations and, potentially, to pay dividends to the financial buyers. Targets usually have low capital expenditures, low existing leverage, and assets that can be sold. An exit event is usually targeted between 3 to 7 years after purchase.

\vspace{5pt}
\noindent The analysis solves for the IRR of the investment, which is the discount rate that results in the cash flow and terminal value of the investment equalling the initial equity investment. If the resulting IRR is below their targeted IRR, the buyer will lower their purchase price. An LBO analysis is similar to a DCF analysis in relation to use of projected cash flows, terminal value, present value and discount rate. The difference is that a DCF analysis solves for the present value while the LBO analysis solves for the discount rate. Once IRR is determined in the LBO analysis, the purchase price may need to increase or decrease in order to align with the targeted IRR.

\vspace{5pt}
\noindent The analysis also considers whether there is enough projected cash flow to operate the company and also pay down debt principal and cover interest payments while paying dividends. The ability to retire debt and pay dividends results in a higher IRR as well. Raising debt and minimizing equity contribution also maximises IRR as IRR is based on the equity involved rather than the total investment sum.

\vspace{10pt}
\noindent \textit{Sum-of-the-Parts Analysis}\\
A breakup analysis is a useful additional valuation tool when a company has many different businesses which, if analysed separately, are worth more than the value of the company as a whole. If the sum of the parts of a company is greater than the current market value of the company, then there may be an opportunity to break up the company and sell it to different buyers, creating incremental value in the sale process. One way is to focus on the parts on a EV/EBITDA multiple bases for each separate business and then add all EVs together to create a case for a higher sale price for the company.

\vspace{5pt}
\noindent Bankers need to determine whether unwanted businesses are best sold in an IPO, carve-out or spin-off, sold to another company or sold to a private equity fund. This will determine the types of valuation methods used referencing the explanations above.

\end{document}