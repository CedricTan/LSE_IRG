\documentclass[a4paper]{article}
\usepackage[margin=1in]{geometry}
\usepackage[bottom]{footmisc}
\usepackage{blindtext}
\usepackage{graphicx}
\graphicspath{ {VC_img/} }


\usepackage{fancyhdr}
\pagestyle{fancy}
\fancyhf{}
\rhead{\leftmark}
\lhead{Venture Capital}

\setcounter{section}{-1}

\title{LSE IRG\\
		Venture Capital}
\author{Cedric Tan - c.tan21@lse.ac.uk}
\date {2019-2020}

\begin{document}
\maketitle
{\small
  \noindent\textbf{Introduction to Venture Capital}\\
  Introduction to the basic terminology of what Venture Capital is and how it fits under Alternative Investments as a private equity investment. What the aims of Venture Capital are and how Venture Capital achieves high returns in the competitive market with so many start-ups present. \hspace*{\fill}[1]

  \vspace{10pt}
  \noindent\textbf{The Start-up Perspective}\\
  Looking at the start-up and how it operates. How they plan to approach the fundraising process and the levels of funding that they are likely to receive at different levels.\hspace*{\fill}[2]
   
  \vspace{10pt}
  \noindent\textbf{Fundraising}\\
  \noindent In depth look into different types of fundraising and types of investors from Angel Investors all the way to the IPO or buyout buy other private equity firms.\hspace*{\fill}[3]

  \vspace{10pt}
  \noindent\textbf{Consulting Frameworks applied to Venture Capital}\\
  Porter's Five Forces, The 3 Cs, The 4 Ps, SWOT analysis along with other frameworks that might be considered when assessing a company's performance and potential for growth.\hspace*{\fill}[4]

  \vspace{10pt}
  \noindent\textbf{Growth Strategies}\\
  How does the company grow alongside venture capital? This section will explore the growth paths of successful start ups under the guidance of big name VCs.  \hspace*{\fill}[5]

  \vspace{10pt}
  \noindent\textbf{Exit Strategies}\\
  We will look at the various ways in which Venture Capital firms plan an exit from their investment. This will mostly cover secondary buyouts and IPOs as exit strategies. \hspace*{\fill}[6]

\newpage
\pagestyle{empty}
\tableofcontents

\newpage
\pagestyle{fancy}

\section{Introduction to Venture Capital}
Below is a brief introduction into Venture Capital. I aim to go into the history of Venture Capital, the very basics of what Venture Capital is and some of the introductory trends in the Venture Capital world.

\subsection{History of Venture Capital}
Venture Capital is a subset of private equity, which can be traced back to the 19th century. However, Venture Capital only developed as an industry after the Second World War. Georges Doriot from Harvard Business School is considered the Father of Venture Capital as he started the American Research and Development Corporation (ARDC) in 1946.

\vspace{5pt}
\noindent He raised a \$3.5 million fund to invest in companies that commercialised technologies developed during WW2. For example, ARDC's first investment was in a company that had ambitiions to use x-ray technology for cancer treatment. The \$200,000 that ARDC invested had a return of \$1.8 million when the company went public in 1955.

\vspace{5pt}
\noindent After the Investment Act of 1958 was passed, access to capital became a lot easier for investment purposes as small business investment companies could be licensed by the Small Business Association. This qualified PE fund managers and provided them access to low-cost, government-guaranteed capital to make investments.

\vspace{5pt}
\noindent It became big in the Bay Area in California where private equity firms in that region set the standards of practice used today. This involved setting up limited partnerships to hold investments where professionals would act as general partners and those supplying the capital would serve as passive partners with more limited control.

\vspace{5pt}
\noindent Venture Capital began posting losses after the 1980s with the most notable period of serious busts occurring during the time frame known as the \textbf{Dot-com Bubble}. This was the time where companies with any semblance of relation to the booming internet were given vast amounts of cash to expand their businesses. Most of these companies ended up failing to make a return for the investor causing serious and heavy losses.

\subsection{Understanding Venture Capitalists}
The basic idea of Venture Capital is to pool money from \textbf{wealthy individuals, pension funds, foundations and corporations} to invest into early-stage start-ups. The money that is pooled is controlled by a Venture Capital Firm that manages and directs investments. All investors (i.e. people who have put in money) have part ownership over the fund \textbf{however, they are limited partners} whilst the firm itself is a \textbf{general partner} which has control over the funds invested.

\vspace{5pt}
\noindent Compensation within Venture Capital follows a similar structure to most Private Equity of Fund Management formats with a \textbf{2 and 20 split} meaning a 2\% annual management fee with a 20\% take on profits for the managing firm.

\vspace{5pt}
\noindent Here is a list of some famous Venture Capital Firms which might sound familiar: \footnote{Find a list of these VCs and their backgrounds in the appendix \textbf{VC Firms}}
\begin{itemize}
	\item Benchmark Capital (AirBnB)
	\item Menlo Capital (Uber)
	\item Andreesen Horowitz (Instagram)
	\item Sequoia Capital (Whatsapp)
	\item Accel Partners (Facebook)
	\item Creandum (Spotify)
\end{itemize}

\subsection{Positions}
Here are some example positions:

\subsubsection{Analyst}
The lowest position at a VC firm, usually graduates out of college or university. Their focus is number-crunching. Analysts screen business plans before passing them to Associates and senior staff and conduct due diligence or research on promising industries and entrepreneurs. Quantitative backgrounds such as Engineering, Math, Economics, or Statistics are also strong backgrounds for the typical Analyst roles.

\subsubsection{Associate}
Associates are usually gate-keepers for top managers of a venture firm. Their primary functions are to source new deals and support existing investments. In large firms, many Associate positions have an ‘analyst’ position below them. Associates also get the chance to network with entrepreneurs and watch the trends within their firm’s industry focus.

\vspace{5pt}
\noindent Traditionally, associates are usually ex-bankers, consultants, investment professionals (i.e. private equity, other VC funds) or operational leaders with three to five years’ experience, sometimes with an MBA or a PhD. However, many top tier venture funds also hire engineers, or operations or product managers with strong technical backgrounds.

\vspace{5pt}
\noindent Commonly, VC firms hire associates for two years only, meaning there is rarely an opportunity for internal advancement at a VC firm. Both senior and junior associates do not usually have the authority to lead investments and sit on boards, thus, they often join a startup in an operating role or go to business school after their contract expires. However, in some firms, an Associate position can lead to a Principal position as noted above.

\subsubsection{Principal}
Principals are typically expected to be a source of high quality deals. Principals are usually full-time employees who are expected to perform such additional functions as investor relations, negotiating partnerships for portfolio companies, management of a fund’s community, etc. Principals at most firms have the authority to lead their own investments and sit on boards, however, they rarely have a vote on deals and only about half of them have direct fund carry, commonly being compensated from the overall performance of the fund.

\subsubsection{Managing Partner} 
Many venture firms call them ‘partners’, some call them ‘managing partners/directors’. You can also see such titles as ’founding’ or ‘co-founding partner’. These are the true GPs of the firm who may also have their names on the door. General partners raise the money for the fund, but also contribute a small amount of their own money to it. They make the final decisions on which companies to invest in and take seats on boards. They hold main stakes in the anticipated carried interest and have the highest compensation at a firm. 

\vspace{5pt}
\noindent At large firms, however, some general partners tend to be less involved in the daily deal-making and are more focused on high-level tasks such as identifying key sectors to invest in, giving the green light for investments and exits, networking at a high level, representing the overall firm, raising money for the next fund and communicating performance to investors.

\subsection{Examples}
Below are some examples of Venture Capitals which have succeeded in great fashion:
\begin{itemize}
	\item Sequoia Capital's \$60m investment into \textbf{Whatsapp} turned into \$3bn after \textbf{3 years} due to the Facebook Acquisition. That's a 50x return.
	\item Accel Partner's \$12.7m investment into \textbf{Facebook} turned into \$9bn after \textbf{7 years} with Facebook's IPO. That's a 708x return.
	\item Creandum's \$4.5m investment into \textbf{Spotify} turned into \$370m after \textbf{11 years} with Spotify's IPO. That's a 80x return.
\end{itemize}
\vspace{5pt}
\noindent In typical investment, returns are calculated on percentage. A standard rate of return for a Private Equity fund would be around 20 to 25\%. For Venture Capitalists, returns are focused on numbers a hundredfold larger. This is why it can be very lucrative.

\subsection{Summary}
To take away, here are the key points of what a Venture Capitalist aims to do:
\begin{itemize}
	\item Find firms which are \textbf{disrupting the market} through their product or service
	\item Invest in these firms to help them grow i.e. capture more of the market, expand internationally etc.
	\item Get a heavy return from their success
\end{itemize}
\vspace{5pt}
Venture Capital is difficult because you're investing in a plethora of firms just to get one hit. However, if that hit is large enough, the returns often offset the other failures as seen by the examples above.

\newpage
\section{Start Up Perspective}
This section looks into the start-up perspective and why Venture Capital is crucial to a lot of start-ups. We will consider some global trends and the type of firms which start-up and require venture capital funding. The key word to recognise for VCs is usually \textbf{disruption} because that means that these firms will be making a \textbf{significant impact} on the market locally and, with expansion, globally.

\subsection{Global View}
Here are some critical focuses that the VC's are currently targeting where about 70\% of VC money is directed:
\begin{itemize}
	\item Software (SaaS, Products)
	\item Biotechnology
	\item Medical Devices
	\item Telecommunications
	\item Hardware e.g. chip-makers
\end{itemize}
\vspace{5pt}
Hence the start-ups in this space are the most popular for VCs to target. This is because they are in a high-growth market with a lot of opportunity to capture market share.\footnote{This is expanded further in \textbf{Consulting Frameworks in VC} through the BCG Growth Matrix}

\subsubsection{Software}
Software is a much broader topic and can cover a lot of bases. To a certain extent, a lot of sharing-economy technology is based on software and has proven to be very disruptive as the traditional model of ownership has been cast aside. Other types of software has also been developed to make processing payments and data much easier. For example, \textbf{Stripe} has made payment processes a lot easier with just a few lines of code.

\vspace{5pt}
\noindent Below are some examples of Software start-ups who are now booming in the industry:
\begin{itemize}
	\item Slack
	\item Asana
	\item Stripe
	\item AirBnB
	\item Spotify
\end{itemize}

\vspace{5pt}
\noindent The list above is far from exhaustive but the brands themselves should be quite indicative of their success as most of them have become household names. This new software has shown a focus on some sort of accessibility. Just going from the examples above, Slack has great messaging facilities and search functionality that allows for easier tracking of records. Asana helps to manage your workflow, Stripe manages payments, AirBnB makes it easier to find a place to stay at cheaper prices with a whole bunch of other options while Spotify makes it easier to listen to music and podcasts.

\vspace{5pt}
\noindent Hence, software such as this, saves precious minutes or just makes it easier for people to do everyday activities. This is what is marketed and it is effective because consumers appreciate the ease of accessibility of these products.

\subsubsection{Biotechnology}

\subsubsection{Medical Devices}

\subsubsection{Telecommuications}


\subsubsection{Hardware}
Hardware is mostly about efficiency whereby we try to increase our levels of output with a smaller amount of resources. Alternatively, hardware is created to make activities much easier or more interactive which brings new types of experiences to consumers. Recent developments in hardware have been dedicated to increase computing power, as seen by the likes of Intel, Nvidia and Qualcomm amongst others, whilst companies such as Fitbit, GoPro and Oculus aim to provide an experience.

\vspace{5pt}
\noindent Below are some examples of Hardware start-ups who are now dominant in the industry:
\begin{itemize}
	\item Nvidia (Sequoia Capital [A]\footnote{Letters indicate the Series round in which they funded the start-up})
	\item Fitbit (Uncork Capital [A] and True Ventures [A])
	\item GoPro (Riverwood Capital [A])
	\item Square (Khosla Ventures [A] and Sequoia Capital [B])
	\item Oculus (Crowdfund, Spark Capital [A] and Matrix Partners [A])
\end{itemize}
\vspace{5pt}
The list goes on. All these firms, stretching from the early 2000s until now, have become household names in their technology. They disrupted the market in some way or form due to their Unique Selling Proposition (USP) founded on making computers faster (Nvidia), making tracking our health easier (Fitbit), capturing our moments with more clarity (GoPro), easing our payments (Square) or even changing our reality (Oculus).

\vspace{5pt}
\noindent Their USP isn't the only reason for their success, many companies have tried the same path before. They were simply better positioned to begin with and managed to convince the VCs they approached that their product was worth selling. They would have had the traction to back it up and the projected finances to show that extreme growth on their end would be possible.

\newpage
\section{Fundraising}
This section will dive into the simple fundraising aspects of venture capital from the VC itself, looking to raise capital to put into start-ups and from the start-up, looking to raise capital at various stages with various investors.

\subsection{VC Funds}
This section will go into VC Funds and how they are raised.

\subsubsection{Structure}
Simply put, institutions with money or high wealth individuals looking to make a return invest their money into a VC's fund. These investors are called \textbf{Limited Partners} as they have a limited say in what the fund does but provides money to gain a return. The fund also has money invested by their \textbf{General Partners} who make the decisions on the investments and control the funds direction.

\vspace{5pt}
\noindent \textbf{The Venture Fund}\\
This is the main investment vehicle used for the actual investing. Each is structured as a limited partnership that is governed by an agreement by the partners for a certain time period. This is usually 7-10 years. It pays out profit sharing through \textbf{carried interest}, which is the share of the effective return on investment, adding to around about 20\% of the funds returns.

\vspace{5pt}
\noindent \textbf{The Management Company}\\
This is the business of the fund which receives a management fee of about 2\% and uses this to pay the overhead related to operating the venture firm. This includes rent, salaries and other operational costs. It makes \textbf{carried interest} only when the limited partners have been repaid.

\vspace{5pt}
\noindent Hence, the \textbf{Venture Fund} invests into \textbf{start-ups} through the funds committed by its limited and general partners.

\subsubsection{Summary}
Each fund is active for 3-4 years i.e. looking for investments in an active manner during this time. However, the venture fund itself has an agreement period or lifetime of 7-10 years. This longer duration is purposed for harvesting returns.

\vspace{5pt}
\noindent A large portion of the fund is saved for follow-on investments for the start-ups within the venture fund portfolio and a firm may have multiple funds running simultaneously but only one usually looking for new investments. 

\subsection{Start-ups}
This section will show some funding stages related to the Start-up Perspective.

\subsubsection{Family and Friends}
Past the ideation and co-founder stage in the start-up, a usual route would be to raise funds from one's own bank account, their friends and family along with potential crowdfunding resources such as Kickstarter.

\subsubsection{Angel and Seed}
Angel and Seed rounds are for investors who take an interest in the minimum viable product and have a belief that it has the potential to disrupt the market drastically. Usually there is no significant track record required for a proper investment but strong growth and expectations for strong growth with at least a market tested product tends to attract more funding.

\vspace{5pt}
\noindent Seed and Angel rounds are usually on the lower end of the investment scale but are still a heavy contributor to the early life of a start-up. Investments range from \$1000s to \$500k in exchange for an equity stake of usually, at the minimum, 20\%.

\subsubsection{Series Funding}
This is where Venture Capital usually lies as the VC firm is there to \textbf{manage risk} alongside entrepreneurs in a \textbf{high-growth set-up} thus an established track record is desired for a proper investment.

\vspace{5pt}
\noindent Series funding starts with a Series A round that can extended as far as \textbf{F} or even further for exceptional cases. Investments in this area are typically on the higher end from \$500k all the way to \$30m and more for an equity stake of usually around 30\%. For example, Uber raised \$3.5bn in a Series G.

\newpage
\section{Consulting Frameworks in VC}
This section is broken into the various frameworks that can be applied when assessing a company and it's competition. Here are some key ones to remember and which will be gone into depth in the section below:
\begin{itemize}
	\item Porter's Five Forces
	\item The 3 Cs
	\item The 4 Ps
	\item SWOT Analysis
\end{itemize}
Other analytical frameworks which are more general can be found in the \textbf{Appendix of Consulting Frameworks.} This includes but is not limited to the Product/Market Grid and the BCG Growth Share Matrix along with some additional analysis concepts such as Break-Evens, Net Profit margins and Return on Investment (ROI).

\vspace{5pt}
\noindent Note that these frameworks are used to structure your analysis and pitches rather than to be stuck to religiously. These are here to help illuminate whether or not a firm has a good product that would be worthwhile investing into.

\subsection{Porter's Five Forces}
This is probably the most known framework used in business strategy worldwide and was created by Harvard Professor Michael Porter. This is a high-level framework i.e. does not dive into the specifics too much, that you can draw upon to perform a market landscape and competitor analysis. 

\vspace{5pt}
\noindent \textit{It can help determine whether a market or company is attractive, whether the client for whom the analysis is being performed is a private equity firm thinking about buying a company, or a major company thinking about entering or exiting a certain market segment.} - Street of Walls

\vspace{5pt}
\noindent Items of analysis are listed below:
\begin{itemize}
	\item Threat of New Entrants
	\item Competitive Dynamics
	\item Supplier Power
	\item Consumer Power
	\item Threat of Substitutes
\end{itemize}

\subsubsection{Threat of New Entrants}
Effectively we can recognise these as barriers to entry i.e. how easy it is for new firms to enter the market and threaten the position the analysed firm is in \textbf{or} how hard it is for the analysed firm to enter the market and whether or not it would make as much of an impact. Below are some key things to consider:

\begin{itemize}
	\item Legal or regulatory barriers
	\item Economies of scale
	\item Cost advantages (not EoS but unique access/deals with raw materials as an example)
	\item Access to distribution channels
	\item Product differentiation
\end{itemize}

\subsubsection{Competitive Dynamics}
This is more suited to understanding the industry landscape and how that might have an impact, whether it be large or small, on the firm. Below are some key things to consider:

\begin{itemize}
	\item Industry growth rate
	\item Industry fragmentation i.e. perfect competition or oligopolistic?
	\item Level of switching costs i.e. changing products from one company to another from the consumer perspective and its impact
	\item Motivation to reduce prices e.g. excess capacity
\end{itemize}

\subsubsection{Supplier Power}
This is more an understanding of how well the supplier is positioned and how much influence it has over the market. Below are some key things to consider:

\begin{itemize}
	\item Level of substitute products
	\item Buyer's decision influenced by supplier
	\item Supplier inputs/products switching costs
	\item Supplier has potential to forward integrate i.e. organic MnA
	\item Supplier accounts for large share of the inputs/products (link back to uniqueness)
\end{itemize}

\subsubsection{Consumer Power}
This is more an understanding of how well the consumer is positioned and how much influence the consumer can have over the market. Below are some key things to consider:

\begin{itemize}
	\item High customer or client concentration
	\item Level of commoditization of inputs/products i.e. treating them as mere commodities or having an added value component attached
	\item Level of switching costs for buyer
	\item Buyer has significant product or market information
\end{itemize}

\subsubsection{Threat of Substitutes}
This is self-explanatory to a large extent. It is a key focus on competition in the current market before new entries or exits i.e. how is the landscape looking \textbf{now} and where the firm is positioned in said landscape (or aims to be). Below are some key things to consider:

\begin{itemize}
	\item Substitute products and services that can compete on price and/or quality
	\item Switch costs to shift to these products
\end{itemize}

\subsection{The 3 Cs}
Quite a lot of overlap with Porter's Five Forces but is shorter and more condensed and still applicable to a wide range of Case Study questions. This one is still good to have in your knowledge base and may cover some points that might be missed in Porter's analysis.

\vspace{5pt}
\noindent Items of analysis are listed below:
\begin{itemize}
	\item Company
	\item Competitors
	\item Customers and Clients
\end{itemize}

\subsubsection{Company}
This is focused on the operations of the company itself and how the company generates revenues. This could include internal operations and best methods in being an efficient company. Below are some key things to consider:

\begin{itemize}
	\item Product or Service offering:
		\begin{itemize} 
			\item What are the pros and cons of this product or service?
			\item What is the value chain
		\end{itemize}
	\item Profitability analysis
	\item Other Company factors:
		\begin{itemize}
			\item Capacity i.e. ability to expand
			\item Core competencies i.e. what is the company focus and how they execute it
			\item Regulatory environment
			\item Distribution network
			\item Management and core employees
		\end{itemize}
\end{itemize}

\subsubsection{Competitors}
This is focused on how competitors impact your client and how the competitive dynamics will change over time. Thus, we can recognise this as a key factor of knowing how to stand out in the market. Below are some key things to consider:

\begin{itemize}
	\item Competitor mix and make-up
		\begin{itemize}
			\item Market Share
			\item Fragmentation
			\item Financial situation e.g. do competitors have deep pockets?
			\item Management
			\item Other competencies e.g. marketing and distribution channels
		\end{itemize}
	\item Competitor products or services
		\begin{itemize}
			\item Value proposition versus client i.e. What is the firm's Unique Selling Proposition (USP)
			\item Value chain i.e. what activities are required to create value
		\end{itemize}
\end{itemize}

\subsubsection{Customers and Clients}
This one is often overlooked because one can be so focused on your competition and how you build a company to fight it yet this is probably the most fundamental aspect. It answers the question of why you're building it which is critical to customer acquisition. Analysing customers and clients is about understanding and knowing them which leads to \textbf{winning} in business. Below are some key things to consider:

\begin{itemize}
	\item Customer mix
		\begin{itemize}
			\item Demographics i.e. age, gender, ethnicity
			\item Values of core customers and clients
			\item Wants and needs of customers and clients
		\end{itemize}
	\item Position with customer and client segments
		\begin{itemize}
			\item Customer and client segment \textbf{sizes}
			\item Customer and client segment \textbf{shares}
			\item Customer and client segment \textbf{growth rate}
		\end{itemize}
	\item Key drives of customer and client decisions
		\begin{itemize}
			\item Price relative to value and market
			\item Product characteristics
			\item Branding
			\item Personnel (especially for B2B businesses)
		\end{itemize}
\end{itemize}

\subsection{The 4 Ps}
This framework is often used specifically whenever there is a marketing component involved in a case e.g. how to increase sales resulting from any profitability optimization case. When combined with the 3 Cs, this framework can cover a lot of topics you will come across. You will recognise overlap between the various frameworks which comes naturally as a result of revisiting the product discussed and so on.

\vspace{5pt}
\noindent Items of analysis are listed below:
\begin{itemize}
	\item Product
	\item Price
	\item Promotion
	\item Placement
\end{itemize}

\subsubsection{Product}
This is looking into the product and/or service and its value proposition. This is what you are selling and it is important to see how a firm stands out in a sea of competitors. Below are some key things to consider:

\begin{itemize}
	\item Company product and/or service qualities, features and attributes: differentiated (unique) or commoditized (uniform)?
	\item Competitor product and/or service qualities, features and attributes
	\item Substitute product options:
		\begin{itemize}
			\item How close are the substitutes
			\item USPs?
			\item Switching costs
		\end{itemize}
	\item Customer value proposition
		\begin{itemize}
			\item Why are customers or clients purchasing the product/service?
			\item Brand, availability, service, value, reliability, aesthetic etc.
		\end{itemize}
\end{itemize}

\subsubsection{Price}
It is critical to understand the company's optimal pricing strategy. For start-ups and firms looking to grow, pricing is likely to change to optimise profitability and success in the market. Below are some key things to consider:

\begin{itemize}
	\item Price elasticity (economics)
		\begin{itemize}
			\item Is the product sufficiently better to justify a higher price?
			\item Customer loyalty and lock-ins
			\item Supply and demand i.e. looking at the state of the market
		\end{itemize}
	\item Price of substitute products and services
	\item Price of competitor products and services i.e. 'perfect substitutes
	\item Market positioning
		\begin{itemize}
			\item Brand position and perception
			\item Status
		\end{itemize}
	\item Profitability i.e. thinking about costs
\end{itemize}

\subsubsection{Promotion}
This is related to marketing strategy and is about reaching and attracting the customer or client. You will find a large overlap with \textbf{Product} as it is focused on the needs of the customers and why the product is made in the first place. This helps determine promotional aspects. Below are some key things to consider:

\begin{itemize}
	\item Which markets and customers should the company target?
	\item Is it successful in reaching this market?
	\item What are the most effective marketing campaign strategies? e.g. A/B testing
	\item Return on marketing spend
	\item Retention rates e.g. if it's a product, they are restocking, if it's a service, they are resubscribing
	\item Can we up-sell or cross-sell? i.e. better products to sell of the same variety or new/different products to sell of the same value
\end{itemize}

\subsubsection{Placement}
This is about getting the products to the customers in a tangible manner and how the company goes about doing this. Below are some key things to consider:

\begin{itemize}
	\item Which distribution channels to use? e.g. selective/exclusive or wide distribution networks?
	\item Transport and logistics
		\begin{itemize}
			\item Seamless delivery? Service online?
			\item Internal transport or outsourced?
		\end{itemize}
	\item Specific locations within the channels
		\item Online marketing, landing pages, Google Adwords
		\item Product placement in stores? (if physical)
\end{itemize}

\subsection{SWOT Analysis}
This type of analysis is more of a mini-framework, specifically for quickly evaluating a single company in an industry. It's far less complete and can miss some important details but gives you a quick and concise way to analyse a company. It's effective at a high-level and is intuitively understood.

\vspace{5pt}
\noindent Items of analysis are listed below:
\begin{itemize}
	\item Strengths: Company strengths within an industry
	\item Weaknesses: Company weaknesses within an industry
	\item Opportunities: What opportunities are available within the industry or, potentially, the opportunity to branch into a new industry
	\item Threats: What threats can hurt company growth or performance such as new entrants, disruptive technology or regulation
\end{itemize}
SWOT analysis is as simple as above and does not dive into too much detail. You can tackle each individual aspect of the SWOT framework with further analysis from frameworks above or those found in the appendix.


\newpage
\section{Growth Strategies}
The Growth perspective from a VC is simply related to helping entrepreneurs and start-ups manage their risk. 

\newpage
\section{Exit Strategies}
\Blindtext

\newpage
\appendix
\section{VC Firms}
Listed here are some of the more famous VC firms and their investing specialities.

\subsection{Accel}
This is an American Venture Capital Firm that works with startups in seed, early and growth-stage investments. This means the seeding rounds and along Series A to D territory. Although their head offices are based in Palo Alto, California, they have funds across Europe, headquartered in London along with offices in India, Australia, Brazil, China and many more.

\vspace{5pt}
\noindent Accel concentrates on the following technology sectors:
\begin{itemize}
	\item Consumer
	\item Infrastructure
	\item Media
	\item Mobile
	\item Software as a Service (SaaS)
	\item Security
	\item Customer Care services
\end{itemize}
\vspace{5pt}
\noindent Some of their recent investments/exits include:
\begin{itemize}
	\item Cloudera with an \textbf{IPO} Valuation of \$2.30bn in 2017
	\item Despegar with an \textbf{IPO} Valuation of \$1.97bn in 2017
	\item Etsy with an \textbf{IPO} Valuation of \$1.78bn in 2015
	\item Jet \textbf{acquired} by Walmart for \$3.3bn in 2016
	\item Krux \textbf{acquired} by Salesforce for \$700m in 2016
	\item Lynda.com \textbf{acquired} by LinkedIn for \$1.5bn in 2015
\end{itemize}

\vspace{5pt}
\noindent Read more about Accel at: https://www.accel.com/

\subsection{Andreessen Horowitz}
Another American VC with its headquarters in Menlo Park. AH Capital Management invests in both early-stage start-ups, in the \$50k range, up to established growth companies raising tens of millions of dollars.

\vspace{5pt}
\noindent AH focuses on:
\begin{itemize}
	\item Mobile
	\item Gaming
	\item Social
	\item E-commerce
	\item Education
	\item Enterprise IT e.g. SaaS, Cloud, Cyber Security
\end{itemize}
\vspace{5pt}
\noindent Some of their notable investments/exits include:
\begin{itemize}
	\item Instagram
	\item Business Insider
	\item Leap Motion
	\item Dollar Shave Club
	\item Github \textbf{acquired} by Microsoft for \$7.5bn in 2018
\end{itemize}

\vspace{5pt}
\noindent Read more about Andreessen Horowitz at: https://a16z.com/

\subsection{Benchmark}
Another American VC with its headquarters in San Francisco, Benchmark primarily focuses on seed money given to start ups. Although their focus is seed, the funding could extend to millions for larger stakes in the company. They were founded in 1995.

\vspace{5pt}
\noindent Benchmark focuses on:
\begin{itemize}
	\item Mobile
	\item Enterprise IT
	\item Social
	\item E-commerce
\end{itemize}
\vspace{5pt}
\noindent Some of their notable investments/exits include:
\begin{itemize}
	\item eBay
	\item Uber
	\item Dropbox
	\item Twitter
	\item Snapchat
	\item Instagram
	\item WeWork
\end{itemize}

\vspace{5pt}
The list goes on for Benchmark who have had over 250 investments in start-ups since its inception in the 90s. Their track record is one of the most successful in the business. Read more about them at: https://www.benchmark.com/

\subsection{Index Ventures}
Index is an international VC with dual headquarters in San Francisco and London. Since its founding in 1996, originally a technology venture arm of a bond-trading firm, Index has raised approximately \$5.6bn for its investments. Their focus is at the seed to series level with a emphasis on growth capital.

\vspace{5pt}
\noindent Index Ventures focuses on:
\begin{itemize}
	\item E-commerce
	\item Fintech
	\item Mobility
	\item Gaming
	\item Infrastructure and AI
	\item Security
\end{itemize}

\vspace{5pt}
\noindent Some of their notable investments/exits include:
\begin{itemize}
	\item Dropbox
	\item Duo Security \textbf{acquired} by Cisco for \$2.35bn
	\item Etsy
	\item Facebook
	\item King (Candy Crush) \textbf{IPO and subsequent acqusition} by Activision Blizzard for \$5.9bn
	\item Sonos
	\item Supercell
\end{itemize}

\vspace{5pt}
\noindent The list also goes on for Index Ventures with a lot of notable companies in their books. Their track record is also very successful. Read more about Index Ventures at: https://www.indexventures.com/

\subsection{Sequoia Capital}
Sequoia Capital is an American VC with headquarters in Menlo Park with a main focus on technology. Sequoia, although its base is in America, has several funds specific to India, Israel and China. Their focus is on seed to growth capital.

\vspace{5pt}
\noindent Sequoia Capital focuses on:
\begin{itemize}
	\item Energy
	\item Fintech
	\item Enterprise Software
	\item Healthcare
	\item Mobile
\end{itemize}

\vspace{5pt}
\noindent Some of their notable investments/exits include:
\begin{itemize}
	\item Apple
	\item Google
	\item Oracle
	\item PayPal
	\item Stripe
	\item YouTube
	\item Instagram
	\item Yahoo
	\item Whatsapp
\end{itemize}

\vspace{5pt}
\noindent Sequoia has a rich history and has invested over 250 companies. They have exited in 68 IPOs and 203 acquisitions as of 2017. Find out more about Sequoia at: https://www.sequoiacap.com/ (cool website!)

\subsection{Bessemer Venture Partners}
Bessemer Venture Partners is a global VC firm with offices in San Francisco, New York City, Boston, Israel and India. They are an established firm, originally a Family Office, only expanding in 1974, after their 1911 founding, to VC. They are focused on seed to growth stages.

\vspace{5pt}
\noindent Bessemer Venture Partners focuses on:
\begin{itemize}
	\item Consumer
	\item Enterprise
	\item Healthcare
\end{itemize}

\vspace{5pt}
\noindent Notable investments/exits include:
\begin{itemize}
	\item Shopify
	\item Yelp
	\item LinkedIn
	\item Skype
	\item LifeLock
	\item Twilio
	\item Wix.com
\end{itemize}

\vspace{5pt}
\noindent They have invested in over 120 IPOs as of 2019. Find out more about them at: https://www.bvp.com/

\subsection{Founders Fund}
Founders Fund is an American VC based in San Francisco founded in 2005. They invest across all stages and sectors. Peter Thiel, known for PayPal, is one of the firms founders and partners.

\vspace{5pt}
\noindent Founders Fund focuses on:
\begin{itemize}
	\item Aerospace
	\item AI
	\item Advanced Computing
	\item Energy
	\item Health
	\item Consumer Internet
\end{itemize}

\vspace{5pt}
\noindent Notable investments/exits include:
\begin{itemize}
	\item AirBnB
	\item Lyft
	\item Spotify
	\item Stripe
	\item Oscar Health
	\item SpaceX
	\item Palantir Technologies
\end{itemize}

\vspace{5pt}
\noindent The unique aspect about the Founders Fund is that it is a generalist firm that invests across all divisions and geographies. Read more about them at: https://foundersfund.com/

\subsection{GGV Capital}
GGV Capital is a global VC that invests in local founders. It's multi-stage but sector focused with an emphasis on seed to growth investments. They have offices in China and America.

\vspace{5pt}
\noindent GGV Capital focuses on:
\begin{itemize}
	\item Consumer and New Retail
	\item Social
	\item Enterprise IT
\end{itemize}

\vspace{5pt}
\noindent Notable investments/exits include:
\begin{itemize}
	\item Alibaba
	\item AthenaHealth
	\item Pandora
	\item Slack
	\item Square
	\item Zendesk
\end{itemize}

\vspace{5pt}
\noindent This VC is more China focused than most. Find out more at: https://www.ggvc.com/

\subsection{IVP}
Also known as Institutional Venture Partners, IVP  is an American VC that invests in later stage ventures. It's one of the older VCs founded in 1980.

\vspace{5pt}
\noindent GGV Capital focuses on:
\begin{itemize}
	\item Consumer and New Retail
	\item Social
	\item Enterprise IT
	\item Education
	\item Media
\end{itemize}

\vspace{5pt}
\noindent Notable investments/exits include:
\begin{itemize}
	\item Twitter
	\item SoundCloud
	\item Indiegogo
	\item Slack
	\item Snapchat
	\item Netflix
\end{itemize}

\vspace{5pt}
\noindent IVP is more focused on the offering stages up to the exit through IPOs or acquisitions. Find out more at: https://www.ivp.com/


\newpage
\section{Consulting Frameworks}
Listed here are further analytical frameworks and concepts that can be used to identify the potential of an investment.

\subsection{BCG Growth-Share Matrix}
This is used mainly to analyse and evaluate product or business lines. It uses a scatter plot with axes of \textbf{market growth} and relative \textbf{market share} to analyse business performance. This is shown below:

\begin{figure}[h]
	\centering
\includegraphics[scale=.17]{bcg-gsm}
	\caption{BCG GSM}
	\label{fig:bcg-gsm}
\end{figure}

\vspace{5pt}
\noindent Within this framework, there are various categories given to each quadrant as seen above. These are:
\begin{itemize}
	\item Cash Cows
	\item Dogs (Pets)
	\item Question Marks
	\item Stars
\end{itemize}

\subsubsection{Cash Cows}
This is where the company has a high market share in a slow-growing industry. These units are usually seen to generate cash in excess of the amount of cash required to maintain the business. This is considered valuable due to the cash generating quality that the firm provides despite them being positioned in a mature market. Investment into these firms would have a high opportunity cost due to them being in an industry with low growth.

\subsubsection{Dogs (Pets)}
These are units with low market share in a mature and slow-growing industry. These are usually break-even businesses which are generating barely enough cash to maintain the business's market share. Such a unit is considered worthless as it is not generating cash for the company. Dogs should be sold off due to how they depress a company's return on assets ratio ($ROA = \frac{Net\; Income}{Average\; Total\; Assets}$).

\subsubsection{Question Marks}
These are businesses operating with a low market share in a high-growth industry. This is the starting point for most businesses as these units have the potential to gain market share and grow into stars and then cash cows when market growth slows. If they do not develop enough and obtain enough market share, they will degrade into dogs when the market declines.

\subsubsection{Stars}
These are units with a high market share in a high-growth industry. They have a niche-leading or market-leading trajectory which means a monopolistic or \textbf{Unique Selling Proposition} with burgeoning/fortuitous proposition drives. \textit{This is related to previous analysis in the consulting frameworks section.} Stars require high amounts of funding to fight competitors and maintain their growth rate. When the industry slows, stars transition into becoming cash cows.

\subsection{Product/Market Grid}
This is used to determine business growth opportunities and it has two dimensions, Product and Market, which determine four growth strategies:

\begin{itemize}
	\item Market Penetration
	\item Market Development
	\item Product Development
	\item Diversification
\end{itemize}
A figure of this is shown below:
\begin{figure}[h]
	\centering
\includegraphics[scale=.25]{ansoff-matrix}
	\caption{Ansoff Matrix}
	\label{fig:ansoff-matrix}
\end{figure}

\subsubsection{Market Penetration}
Company strategies based on market penetration normally focus on changing incidental clients to regular clients i.e. one offs to subscribers and regular clients to heavy clients i.e. subscribers to premium subscribers. Typical systems in which they can do this are volume discounts, bonus cards, loyalty cards and customer relationship management

\subsubsection{Market Development}
Strategies involving development try to lure clients away from competitors or introduce existing products in foreign markets or introduce new brand names into a market. This is about developing the business within the market to capture further market share.

\subsubsection{Product Development}
Company strategies based on product development often try to sell other products to clients. This is developing the product proposition of the business, up-selling and cross-selling with existing clientele. This can be accessories, add-ons or completely new products. Often existing communication channels are leveraged.

\subsubsection{Diversification}
Company strategies based on diversification are the most risky type of strategy. This is the business taking a new direction into a new market and landscape. We can also recognise that diversification can take various forms:
\begin{itemize}
	\item Horizontal: new product with a current market
	\item Vertical: integration of suppliers or customer businesses
	\item Concentric: new product closely related to current product in a new market
	\item Conglomerate: new product in new market
\end{itemize}

\subsection{Break-Even Analysis}
This is when the number of units sold generates revenues equal to total expenses (Fixed Expenses plus Variable Expenses). This type of analysis is often applied when deciding whether to develop a new product or make a capital equipment investment. It also aids in deciding how to price products and services and the number of products to produce.

\subsection{Net Profit margin}
This is referring to the total Net Income of a company or business as a percentage of its revenue i.e. $NPR = \frac{NI}{TR}$. Other variations could include \textbf{Gross Profit Margin} with the gross profit being the numerator and \textbf{Operating Profit Margin (EBIT Margin)} or \textbf{EBITDA Margin} which simply replaces the numerator for purposes of analysis.

\subsection{Return on Investment}
This is the ratio that determines the return or Profit from capital invested. This is to assess the feasibility of a potential investment or acquisition and is used heavily in finance related to MnA, PE and VC.

\vspace{5pt}
\noindent Standard ROI is calculated as follows: $ROI = \frac{R-C}{Capital}$. Return on Assets (ROA) is a variation of this concept but revolves around all capital invested i.e. Liabilities and Equity rather than equity alone.

\subsection{Compound Annual growth Rate (CAGR)}
This is the percentage rate at which any figure e.g. units sold, population or investment, must grow in each year to reach a given end value over a certain amount of time. This is similar to the Internal Rate of Return (IRR) which is the annual rate of return on an investment if its value grows by a specific multiple over a specific amount of time.

\vspace{5pt}
\noindent The formula for CAGR is: $[(\frac{Ending\; Value}{Beginning\; Value})^{\frac{1}{Years}}]-1$

\subsection{Lifetime Customer Value (LCV)}
This projects the total profitability attributed to a firm's future relationship to a typical customer. This is to determine the reasonable cost to win or acquire a customer. In can also help determine the value of a business: $LCV \times Customers + growth\; opportunities$.

\vspace{5pt}
\noindent Steps to calculating LCV are as follows:
\begin{enumerate}
	\item Estimate the remaining customers years i.e. how long a customer stays with the company
	\item Estimate future revenue per year per customers i.e. product volume per customer times prices
	\item Estimate total expenses for producing those products
	\item Calculate Net Present Value of the future profit per customer
\end{enumerate}

\subsection{Product Life Cycle}
This is important for market sizing problems. This is to help project company's to project their own anticipated revenue figures. It is good to know the \textbf{Product Life Cycle Curve:}
\begin{itemize}
	\item Emerging: a new product or technology that is in initial adoption phases and therefore has very rapid growth rates
	\item Growth: product adoption is becoming widespread but still growing at an above-average rate
	\item Maturity: product adoption is widespread or at least stabilised; growth typically comes only from price increases and growth in GDP
	\item Declining: technological obsolescence, shifting consumption patterns or increased market competition has resulted in total growth rates that are below-average or negative
\end{itemize}

\end{document}