\documentclass[a4paper]{article}
\usepackage[margin=1in]{geometry}
\usepackage{blindtext}
\usepackage{graphicx}
\graphicspath{ {VC_img/} }


\usepackage{fancyhdr}
\pagestyle{fancy}
\fancyhf{}
\rhead{\leftmark}
\lhead{Venture Capital}

\setcounter{section}{-1}

\title{LSE IRG\\
		Venture Capital}
\author{Cedric Tan - c.tan21@lse.ac.uk}
\date {2019-2020}

\begin{document}
\maketitle
{\small
  \noindent\textbf{Introduction to Venture Capital}\\
  Introduction to the basic terminology of what Venture Capital is and how it fits under Alternative Investments as a private equity investment. What the aims of Venture Capital are and how Venture Capital achieves high returns in the competitive market with so many start-ups present. \hspace*{\fill}[1]

  \vspace{10pt}
  \noindent\textbf{The Start-up Perspective}\\
  Looking at the start-up and how it operates. How they plan to approach the fundraising process and the levels of funding that they are likely to receive at different levels.\hspace*{\fill}[2]
   
  \vspace{10pt}
  \noindent\textbf{Fundraising}\\
  \noindent In depth look into different types of fundraising and types of investors from Angel Investors all the way to the IPO or buyout buy other private equity firms.\hspace*{\fill}[3]

  \vspace{10pt}
  \noindent\textbf{Consulting Frameworks applied to Venture Capital}\\
  Porter's Five Forces, The 3 Cs, The 4 Ps, SWOT analysis along with other frameworks that might be considered when assessing a company's performance and potential for growth.\hspace*{\fill}[4]

  \vspace{10pt}
  \noindent\textbf{Growth Strategies}\\
  How does the company grow alongside venture capital? This section will explore the growth paths of successful start ups under the guidance of big name VCs.  \hspace*{\fill}[5]

  \vspace{10pt}
  \noindent\textbf{Exit Strategies}\\
  We will look at the various ways in which Venture Capital firms plan an exit from their investment. This will mostly cover secondary buyouts and IPOs as exit strategies. \hspace*{\fill}[6]

\newpage
\pagestyle{empty}
\tableofcontents

\newpage
\pagestyle{fancy}

\section{Introduction to Venture Capital}
Below is a brief introduction into Venture Capital. I aim to go into the history of Venture Capital, the very basics of what Venture Capital is and some of the introductory trends in the Venture Capital world.

\subsection{History of Venture Capital}
Venture Capital is a subset of private equity, which can be traced back to the 19th century. However, Venture Capital only developed as an industry after the Second World War. Georges Doriot from Harvard Business School is considered the Father of Venture Capital as he started the American Research and Development Corporation (ARDC) in 1946.

\vspace{5pt}
\noindent He raised a \$3.5 million fund to invest in companies that commercialised technologies developed during WW2. For example, ARDC's first investment was in a company that had ambitiions to use x-ray technology for cancer treatment. The \$200,000 that ARDC invested had a return of \$1.8 million when the company went public in 1955.

\vspace{5pt}
\noindent After the Investment Act of 1958 was passed, access to capital became a lot easier for investment purposes as small business investment companies could be licensed by the Small Business Association. This qualified PE fund managers and provided them access to low-cost, government-guaranteed capital to make investments.

\vspace{5pt}
\noindent It became big in the Bay Area in California where private equity firms in that region set the standards of practice used today. This involved setting up limited partnerships to hold investments where professionals would act as general partners and those supplying the capital would serve as passive partners with more limited control.

\subsection{Understanding Venture Capitalists}

\newpage
\section{Start Up Perspective}
This section looks into the start-up perspective and why Venture Capital is crucial to a lot of start-ups. We will consider our world as it is, the type of firms which start-up and don't require venture capital funding and the firms which do.

\newpage
\section{Fundraising}
This section will dive into the fundraising aspects of venture capital from the VC itself, looking to raise capital to put into start-ups and from the start-up, looking to raise capital at various stages with various investors.



\newpage
\section{Consulting Frameworks in VC}
This section is broken into the various frameworks that can be applied when assessing a company and it's competition. Here are some key ones to remember and which will be gone into depth in the section below:
\begin{itemize}
	\item Porter's Five Forces
	\item The 3 Cs
	\item The 4 Ps
	\item SWOT Analysis
\end{itemize}
Other analytical frameworks which are more general can be found in the \textbf{Appendix of Consulting Frameworks.} This includes but is not limited to the Product/Market Grid and the BCG Growth Share Matrix along with some additional analysis concepts such as Break-Evens, Net Profit margins and Return on Investment (ROI).

\vspace{5pt}
\noindent Note that these frameworks are used to structure your analysis and pitches rather than to be stuck to religiously. These are here to help illuminate whether or not a firm has a good product that would be worthwhile investing into.

\subsection{Porter's Five Forces}
This is probably the most known framework used in business strategy worldwide and was created by Harvard Professor Michael Porter. This is a high-level framework i.e. does not dive into the specifics too much, that you can draw upon to perform a market landscape and competitor analysis. 

\vspace{5pt}
\noindent \textit{It can help determine whether a market or company is attractive, whether the client for whom the analysis is being performed is a private equity firm thinking about buying a company, or a major company thinking about entering or exiting a certain market segment.} - Street of Walls

\vspace{5pt}
\noindent Items of analysis are listed below:
\begin{itemize}
	\item Threat of New Entrants
	\item Competitive Dynamics
	\item Supplier Power
	\item Consumer Power
	\item Threat of Substitutes
\end{itemize}

\subsubsection{Threat of New Entrants}
Effectively we can recognise these as barriers to entry i.e. how easy it is for new firms to enter the market and threaten the position the analysed firm is in \textbf{or} how hard it is for the analysed firm to enter the market and whether or not it would make as much of an impact. Below are some key things to consider:

\begin{itemize}
	\item Legal or regulatory barriers
	\item Economies of scale
	\item Cost advantages (not EoS but unique access/deals with raw materials as an example)
	\item Access to distribution channels
	\item Product differentiation
\end{itemize}

\subsubsection{Competitive Dynamics}
This is more suited to understanding the industry landscape and how that might have an impact, whether it be large or small, on the firm. Below are some key things to consider:

\begin{itemize}
	\item Industry growth rate
	\item Industry fragmentation i.e. perfect competition or oligopolistic?
	\item Level of switching costs i.e. changing products from one company to another from the consumer perspective and its impact
	\item Motivation to reduce prices e.g. excess capacity
\end{itemize}

\subsubsection{Supplier Power}
This is more an understanding of how well the supplier is positioned and how much influence it has over the market. Below are some key things to consider:

\begin{itemize}
	\item Level of substitute products
	\item Buyer's decision influenced by supplier
	\item Supplier inputs/products switching costs
	\item Supplier has potential to forward integrate i.e. organic MnA
	\item Supplier accounts for large share of the inputs/products (link back to uniqueness)
\end{itemize}

\subsubsection{Consumer Power}
This is more an understanding of how well the consumer is positioned and how much influence the consumer can have over the market. Below are some key things to consider:

\begin{itemize}
	\item High customer or client concentration
	\item Level of commoditization of inputs/products i.e. treating them as mere commodities or having an added value component attached
	\item Level of switching costs for buyer
	\item Buyer has significant product or market information
\end{itemize}

\subsubsection{Threat of Substitutes}
This is self-explanatory to a large extent. It is a key focus on competition in the current market before new entries or exits i.e. how is the landscape looking \textbf{now} and where the firm is positioned in said landscape (or aims to be). Below are some key things to consider:

\begin{itemize}
	\item Substitute products and services that can compete on price and/or quality
	\item Switch costs to shift to these products
\end{itemize}

\subsection{The 3 Cs}
Quite a lot of overlap with Porter's Five Forces but is shorter and more condensed and still applicable to a wide range of Case Study questions. This one is still good to have in your knowledge base and may cover some points that might be missed in Porter's analysis.

\vspace{5pt}
\noindent Items of analysis are listed below:
\begin{itemize}
	\item Company
	\item Competitors
	\item Customers and Clients
\end{itemize}

\subsubsection{Company}
This is focused on the operations of the company itself and how the company generates revenues. This could include internal operations and best methods in being an efficient company. Below are some key things to consider:

\begin{itemize}
	\item Product or Service offering:
		\begin{itemize} 
			\item What are the pros and cons of this product or service?
			\item What is the value chain
		\end{itemize}
	\item Profitability analysis
	\item Other Company factors:
		\begin{itemize}
			\item Capacity i.e. ability to expand
			\item Core competencies i.e. what is the company focus and how they execute it
			\item Regulatory environment
			\item Distribution network
			\item Management and core employees
		\end{itemize}
\end{itemize}

\subsubsection{Competitors}
This is focused on how competitors impact your client and how the competitive dynamics will change over time. Thus, we can recognise this as a key factor of knowing how to stand out in the market. Below are some key things to consider:

\begin{itemize}
	\item Competitor mix and make-up
		\begin{itemize}
			\item Market Share
			\item Fragmentation
			\item Financial situation e.g. do competitors have deep pockets?
			\item Management
			\item Other competencies e.g. marketing and distribution channels
		\end{itemize}
	\item Competitor products or services
		\begin{itemize}
			\item Value proposition versus client i.e. What is the firm's Unique Selling Proposition (USP)
			\item Value chain i.e. what activities are required to create value
		\end{itemize}
\end{itemize}

\subsubsection{Customers and Clients}
This one is often overlooked because one can be so focused on your competition and how you build a company to fight it yet this is probably the most fundamental aspect. It answers the question of why you're building it which is critical to customer acquisition. Analysing customers and clients is about understanding and knowing them which leads to \textbf{winning} in business. Below are some key things to consider:

\begin{itemize}
	\item Customer mix
		\begin{itemize}
			\item Demographics i.e. age, gender, ethnicity
			\item Values of core customers and clients
			\item Wants and needs of customers and clients
		\end{itemize}
	\item Position with customer and client segments
		\begin{itemize}
			\item Customer and client segment \textbf{sizes}
			\item Customer and client segment \textbf{shares}
			\item Customer and client segment \textbf{growth rate}
		\end{itemize}
	\item Key drives of customer and client decisions
		\begin{itemize}
			\item Price relative to value and market
			\item Product characteristics
			\item Branding
			\item Personnel (especially for B2B businesses)
		\end{itemize}
\end{itemize}

\subsection{The 4 Ps}
This framework is often used specifically whenever there is a marketing component involved in a case e.g. how to increase sales resulting from any profitability optimization case. When combined with the 3 Cs, this framework can cover a lot of topics you will come across. You will recognise overlap between the various frameworks which comes naturally as a result of revisiting the product discussed and so on.

\vspace{5pt}
\noindent Items of analysis are listed below:
\begin{itemize}
	\item Product
	\item Price
	\item Promotion
	\item Placement
\end{itemize}

\subsubsection{Product}
This is looking into the product and/or service and its value proposition. This is what you are selling and it is important to see how a firm stands out in a sea of competitors. Below are some key things to consider:

\begin{itemize}
	\item Company product and/or service qualities, features and attributes: differentiated (unique) or commoditized (uniform)?
	\item Competitor product and/or service qualities, features and attributes
	\item Substitute product options:
		\begin{itemize}
			\item How close are the substitutes
			\item USPs?
			\item Switching costs
		\end{itemize}
	\item Customer value proposition
		\begin{itemize}
			\item Why are customers or clients purchasing the product/service?
			\item Brand, availability, service, value, reliability, aesthetic etc.
		\end{itemize}
\end{itemize}

\subsubsection{Price}
It is critical to understand the company's optimal pricing strategy. For start-ups and firms looking to grow, pricing is likely to change to optimise profitability and success in the market. Below are some key things to consider:

\begin{itemize}
	\item Price elasticity (economics)
		\begin{itemize}
			\item Is the product sufficiently better to justify a higher price?
			\item Customer loyalty and lock-ins
			\item Supply and demand i.e. looking at the state of the market
		\end{itemize}
	\item Price of substitute products and services
	\item Price of competitor products and services i.e. 'perfect substitutes
	\item Market positioning
		\begin{itemize}
			\item Brand position and perception
			\item Status
		\end{itemize}
	\item Profitability i.e. thinking about costs
\end{itemize}

\subsubsection{Promotion}
This is related to marketing strategy and is about reaching and attracting the customer or client. You will find a large overlap with \textbf{Product} as it is focused on the needs of the customers and why the product is made in the first place. This helps determine promotional aspects. Below are some key things to consider:

\begin{itemize}
	\item Which markets and customers should the company target?
	\item Is it successful in reaching this market?
	\item What are the most effective marketing campaign strategies? e.g. A/B testing
	\item Return on marketing spend
	\item Retention rates e.g. if it's a product, they are restocking, if it's a service, they are resubscribing
	\item Can we up-sell or cross-sell? i.e. better products to sell of the same variety or new/different products to sell of the same value
\end{itemize}

\subsubsection{Placement}
This is about getting the products to the customers in a tangible manner and how the company goes about doing this. Below are some key things to consider:

\begin{itemize}
	\item Which distribution channels to use? e.g. selective/exclusive or wide distribution networks?
	\item Transport and logistics
		\begin{itemize}
			\item Seamless delivery? Service online?
			\item Internal transport or outsourced?
		\end{itemize}
	\item Specific locations within the channels
		\item Online marketing, landing pages, Google Adwords
		\item Product placement in stores? (if physical)
\end{itemize}

\subsection{SWOT Analysis}
This type of analysis is more of a mini-framework, specifically for quickly evaluating a single company in an industry. It's far less complete and can miss some important details but gives you a quick and concise way to analyse a company. It's effective at a high-level and is intuitively understood.

\vspace{5pt}
\noindent Items of analysis are listed below:
\begin{itemize}
	\item Strengths: Company strengths within an industry
	\item Weaknesses: Company weaknesses within an industry
	\item Opportunities: What opportunities are available within the industry or, potentially, the opportunity to branch into a new industry
	\item Threats: What threats can hurt company growth or performance such as new entrants, disruptive technology or regulation
\end{itemize}
SWOT analysis is as simple as above and does not dive into too much detail. You can tackle each individual aspect of the SWOT framework with further analysis from frameworks above or those found in the appendix.


\newpage
\section{Growth Strategies}
\Blindtext

\newpage
\section{Exit Strategies}
\Blindtext

\newpage
\appendix
\section{Consulting Frameworks}
Listed here are further analytical frameworks and concepts that can be used to identify the potential of an investment.

\subsection{BCG Growth-Share Matrix}
This is used mainly to analyse and evaluate product or business lines. It uses a scatter plot with axes of \textbf{market growth} and relative \textbf{market share} to analyse business performance. This is shown below:

\begin{figure}[h]
	\centering
\includegraphics[scale=.17]{bcg-gsm}
	\caption{BCG GSM}
	\label{fig:bcg-gsm}
\end{figure}

\vspace{5pt}
\noindent Within this framework, there are various categories given to each quadrant as seen above. These are:
\begin{itemize}
	\item Cash Cows
	\item Dogs (Pets)
	\item Question Marks
	\item Stars
\end{itemize}

\subsubsection{Cash Cows}
This is where the company has a high market share in a slow-growing industry. These units are usually seen to generate cash in excess of the amount of cash required to maintain the business. This is considered valuable due to the cash generating quality that the firm provides despite them being positioned in a mature market. Investment into these firms would have a high opportunity cost due to them being in an industry with low growth.

\subsubsection{Dogs (Pets)}
These are units with low market share in a mature and slow-growing industry. These are usually break-even businesses which are generating barely enough cash to maintain the business's market share. Such a unit is considered worthless as it is not generating cash for the company. Dogs should be sold off due to how they depress a company's return on assets ratio ($ROA = \frac{Net\; Income}{Average\; Total\; Assets}$).

\subsubsection{Question Marks}
These are businesses operating with a low market share in a high-growth industry. This is the starting point for most businesses as these units have the potential to gain market share and grow into stars and then cash cows when market growth slows. If they do not develop enough and obtain enough market share, they will degrade into dogs when the market declines.

\subsubsection{Stars}
These are units with a high market share in a high-growth industry. They have a niche-leading or market-leading trajectory which means a monopolistic or \textbf{Unique Selling Proposition} with burgeoning/fortuitous proposition drives. \textit{This is related to previous analysis in the consulting frameworks section.} Stars require high amounts of funding to fight competitors and maintain their growth rate. When the industry slows, stars transition into becoming cash cows.

\subsection{Product/Market Grid}
This is used to determine business growth opportunities and it has two dimensions, Product and Market, which determine four growth strategies:

\begin{itemize}
	\item Market Penetration
	\item Market Development
	\item Product Development
	\item Diversification
\end{itemize}
A figure of this is shown below:
\begin{figure}[h]
	\centering
\includegraphics[scale=.25]{ansoff-matrix}
	\caption{Ansoff Matrix}
	\label{fig:ansoff-matrix}
\end{figure}

\subsubsection{Market Penetration}
Company strategies based on market penetration normally focus on changing incidental clients to regular clients i.e. one offs to subscribers and regular clients to heavy clients i.e. subscribers to premium subscribers. Typical systems in which they can do this are volume discounts, bonus cards, loyalty cards and customer relationship management

\subsubsection{Market Development}
Strategies involving development try to lure clients away from competitors or introduce existing products in foreign markets or introduce new brand names into a market. This is about developing the business within the market to capture further market share.

\subsubsection{Product Development}
Company strategies based on product development often try to sell other products to clients. This is developing the product proposition of the business, up-selling and cross-selling with existing clientele. This can be accessories, add-ons or completely new products. Often existing communication channels are leveraged.

\subsubsection{Diversification}
Company strategies based on diversification are the most risky type of strategy. This is the business taking a new direction into a new market and landscape. We can also recognise that diversification can take various forms:
\begin{itemize}
	\item Horizontal: new product with a current market
	\item Vertical: integration of suppliers or customer businesses
	\item Concentric: new product closely related to current product in a new market
	\item Conglomerate: new product in new market
\end{itemize}

\subsection{Break-Even Analysis}
This is when the number of units sold generates revenues equal to total expenses (Fixed Expenses plus Variable Expenses). This type of analysis is often applied when deciding whether to develop a new product or make a capital equipment investment. It also aids in deciding how to price products and services and the number of products to produce.

\subsection{Net Profit margin}
This is referring to the total Net Income of a company or business as a percentage of its revenue i.e. $NPR = \frac{NI}{TR}$. Other variations could include \textbf{Gross Profit Margin} with the gross profit being the numerator and \textbf{Operating Profit Margin (EBIT Margin)} or \textbf{EBITDA Margin} which simply replaces the numerator for purposes of analysis.

\subsection{Return on Investment}
This is the ratio that determines the return or Profit from capital invested. This is to assess the feasibility of a potential investment or acquisition and is used heavily in finance related to MnA, PE and VC.

\vspace{5pt}
\noindent Standard ROI is calculated as follows: $ROI = \frac{R-C}{Capital}$. Return on Assets (ROA) is a variation of this concept but revolves around all capital invested i.e. Liabilities and Equity rather than equity alone.

\subsection{Compound Annual growth Rate (CAGR)}
This is the percentage rate at which any figure e.g. units sold, population or investment, must grow in each year to reach a given end value over a certain amount of time. This is similar to the Internal Rate of Return (IRR) which is the annual rate of return on an investment if its value grows by a specific multiple over a specific amount of time.

\vspace{5pt}
\noindent The formula for CAGR is: $[(\frac{Ending\; Value}{Beginning\; Value})^{\frac{1}{Years}}]-1$

\subsection{Lifetime Customer Value (LCV)}
This projects the total profitability attributed to a firm's future relationship to a typical customer. This is to determine the reasonable cost to win or acquire a customer. In can also help determine the value of a business: $LCV \times Customers + growth\; opportunities$.

\vspace{5pt}
\noindent Steps to calculating LCV are as follows:
\begin{enumerate}
	\item Estimate the remaining customers years i.e. how long a customer stays with the company
	\item Estimate future revenue per year per customers i.e. product volume per customer times prices
	\item Estimate total expenses for producing those products
	\item Calculate Net Present Value of the future profit per customer
\end{enumerate}

\subsection{Product Life Cycle}
This is important for market sizing problems. This is to help project company's to project their own anticipated revenue figures. It is good to know the \textbf{Product Life Cycle Curve:}
\begin{itemize}
	\item Emerging: a new product or technology that is in initial adoption phases and therefore has very rapid growth rates
	\item Growth: product adoption is becoming widespread but still growing at an above-average rate
	\item Maturity: product adoption is widespread or at least stabilised; growth typically comes only from price increases and growth in GDP
	\item Declining: technological obsolescence, shifting consumption patterns or increased market competition has resulted in total growth rates that are below-average or negative
\end{itemize}

\end{document}