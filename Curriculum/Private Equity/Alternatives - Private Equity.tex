\documentclass[a4paper]{article}
\usepackage[margin=1in]{geometry}
\usepackage{blindtext}

\usepackage{fancyhdr}
\pagestyle{fancy}
\fancyhf{}
\rhead{\leftmark}
\lhead{Private Equity}

\setcounter{section}{-1}

\title{LSE IRG\\
		Private Equity}
\author{Cedric Tan - c.tan21@lse.ac.uk}
\date {2019-2020}

\begin{document}
\maketitle
{\small
  \noindent\textbf{Introduction to Private Equity}\\
  filler \hspace*{\fill}[1]

  \vspace{10pt}
  \noindent\textbf{The Fund Perspective}\\
  We will go into what a fund looks like in terms of structure along with examples of some of the most famous funds in the world. Moreover, we will discuss fund objectives and terminology relating to how a fund tracks its returns.\hspace*{\fill}[2]
  
  \vspace{5pt}
  \noindent We will also explore the fund-raising process and the vehicles set up to conduct the process within Private Equity by exploring funds that have been raised for notable PE firms such as KKR and Blackstone. \hspace*{\fill}[3]

  \vspace{10pt}
  \noindent\textbf{Acquisitions}\\
  We will look into how a PE firm scouts and looks for an investment before they drill down to a few selections through a holistic overview of their Due Diligence process. We will go through the Offer Memorandum process and check out what would be inside those, NDAs and their relevance, Letters of Intent, Preliminary Investment Memorandum (PIM), operating models and the Final Investment Memorandum (FIM) before the offer.  \hspace*{\fill}[4]
  
  \vspace{10pt}
  \noindent\textbf{LBO Modelling}\\
  We will look into a simple example of an LBO model that goes to show the various ways in which a company may acquire another company. This will be done through an Excel model with annotated steps. \hspace*{\fill}[5]

  \vspace{10pt}
  \noindent\textbf{Growth Strategies}\\
  Looking at how Private Equity companies make their investments lean and mean. We will go over some critical factors such as the Internal Rate of Return (IRR), Earnings before Interest, Tax, Depreciation and Amortisation (EBITDA) along with Realization Multiples. \hspace*{\fill}[6]

  \vspace{10pt}
  \noindent\textbf{Exit Strategies}\\
  Looking at the ways in which PE firms exit through Initial Public Offerings (IPOs), secondary buyouts and in the worst of cases, liquidations.  \hspace*{\fill}[7]

  

\newpage
\pagestyle{empty}
\tableofcontents

\newpage
\pagestyle{fancy}

\end{document}