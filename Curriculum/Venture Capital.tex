\documentclass[a4paper]{article}
\usepackage[margin=1in]{geometry}
\usepackage{blindtext}

\usepackage{fancyhdr}
\pagestyle{fancy}
\fancyhf{}
\rhead{\leftmark}
\lhead{Venture Capital}

\setcounter{section}{-1}

\title{LSE IRG\\
		Venture Capital}
\author{Cedric Tan - c.tan21@lse.ac.uk}
\date {2019-2020}

\begin{document}
\maketitle
{\small
  \noindent\textbf{Introduction to Venture Capital}\\
  Introduction to the basic terminology of what Venture Capital is and how it fits under Alternative Investments as a private equity investment. What the aims of Venture Capital are and how Venture Capital achieves high returns in the competitive market with so many start-ups present. \hspace*{\fill}[1]

  \vspace{10pt}
  \noindent\textbf{The Start-up Perspective}\\
  Looking at the start-up and how it operates. How they plan to approach the fundraising process and the levels of funding that they are likely to receive at different levels.\hspace*{\fill}[2]
   
  \vspace{10pt}
  \noindent\textbf{Fundraising}\\
  \noindent In depth look into different types of fundraising and types of investors from Angel Investors all the way to the IPO or buyout buy other private equity firms.\hspace*{\fill}[3]

  \vspace{10pt}
  \noindent\textbf{Consulting Frameworks applied to Venture Capital}\\
  Porter's Five Forces, The 3 Cs, The 4 Ps, SWOT analysis along with other frameworks that might be considered when assessing a company's performance and potential for growth.\hspace*{\fill}[4]

  \vspace{10pt}
  \noindent\textbf{Growth Strategies}\\
  How does the company grow alongside venture capital? This section will explore the growth paths of successful start ups under the guidance of big name VCs.  \hspace*{\fill}[5]

  \vspace{10pt}
  \noindent\textbf{Exit Strategies}\\
  We will look at the various ways in which Venture Capital firms plan an exit from their investment. This will mostly cover secondary buyouts and IPOs as exit strategies. \hspace*{\fill}[6]

\newpage
\pagestyle{empty}
\tableofcontents

\newpage
\pagestyle{fancy}

\section{Introduction to Venture Capital}
Below is a brief introduction into Venture Capital. I aim to go into the history of Venture Capital, the very basics of what Venture Capital is and some of the introductory trends in the Venture Capital world.

\subsection{History of Venture Capital}
Venture Capital is a subset of private equity, which can be traced back to the 19th century. However, Venture Capital only developed as an industry after the Second World War. Georges Doriot from Harvard Business School is considered the Father of Venture Capital as he started the American Research and Development Corporation (ARDC) in 1946.

He raised a \$3.5 million fund to invest in companies that commercialised technologies developed during WW2. For example, ARDC's first investment was in a company that had ambitiions to use x-ray technology for cancer treatment. The \$200,000 that ARDC invested had a return of \$1.8 million when the company went public in 1955.

\newpage
\section{Start Up Perspective}
This section looks into the start-up perspective and why Venture Capital is crucial to a lot of start-ups. We will consider our world as it is, the type of firms which start-up and don't require venture capital funding and the firms which do.

\newpage
\section{Fundraising}
\Blindtext

\newpage
\section{Consultuing Frameworks in VC}
\Blindtext

\newpage
\section{Growth Strategies}
\Blindtext

\newpage
\section{Exit Strategies}
\Blindtext

\end{document}